\documentclass[border=2mm]{standalone}
\usepackage{pgfplots}
\usepackage{amsmath}
\usepackage[scaled]{helvet}
\usepackage[T1]{fontenc}
\renewcommand\familydefault{\sfdefault}
\usepackage[eulergreek]{sansmath}
\pgfplotsset{
tick label style = {font=\sansmath\sffamily}}

\pgfkeys{/pgf/number format/fixed, /pgf/number format/precision=5, /pgf/number format/.cd, 1000 sep = {}}
\pgfplotsset{compat=newest}

\begin{document}

\pgfmathdeclarefunction{gauss}{2}{\pgfmathparse{1/(#2*sqrt(2*pi))*exp(-((x-#1)^2)/(2*#2^2))}%
}

\begin{tikzpicture}[
    declare function={gamma(\z)=
    2.506628274631*sqrt(1/\z)+ 0.20888568*(1/\z)^(1.5)+ 0.00870357*(1/\z)^(2.5)- (174.2106599*(1/\z)^(3.5))/25920- (715.6423511*(1/\z)^(4.5))/1244160)*exp((-ln(1/\z)-1)*\z;},
    declare function={student(\x,\n)= gamma((\n+1)/2.)/(sqrt(\n*pi) *gamma(\n/2.)) *((1+(\x*\x)/\n)^(-(\n+1)/2.));}
]
\begin{axis}[
    width=10cm, % Set the width of the axis
    height=6cm, % Set the height of the axis
    scaled ticks=false,
    axis lines=center,
    enlargelimits=upper,
    line width=0.8,
    xlabel = {$x$},
    ylabel = {$f(x)$}, 
    yticklabel = \empty,
    xmin = -5, xmax = 4.5,
    ymin = -0.05, ymax = 0.45,
    every axis x label/.style={at={(ticklabel* cs:1.02)}, anchor=west,}, every axis y label/.style={at={(ticklabel* cs:1.02)}, anchor=south}
]
\addplot + [smooth, mark=none, domain=-4:4, samples=100, color=red, line width=1.5] {gauss(0,1)};

\pgfplotsinvokeforeach{1}{
    \addplot [thick, smooth, domain=-5:5, samples=100, color=blue, line width=1.2] {student(x,#1)};
}
\node at (2.5,0.2) {\color{blue}$\nu=1$};
\pgfplotsinvokeforeach{5}{
    \addplot [thick, smooth, domain=-5:5, samples=100, color=cyan, line width=1.2] {student(x,#1)};
}
\node at (2.5,0.25) {\color{cyan}$\nu=5$};
\pgfplotsinvokeforeach{10}{
    \addplot [thick, smooth, domain=-5:5, samples=100, color=orange, line width=1.2] {student(x,#1)};
}
\node at (2.5,0.3) {\color{orange}$\nu=10$};
\node at (2.5,0.35) {\color{red}$\nu\to+\infty$};
\node at (2.5,0.4) {\footnotesize\color{gray}Normal distribution};

\draw[arrows=->, thick, color=blue!20] (-1,0.41)--(-0.1,0.32);
\node at (-1.5,0.44) {\footnotesize\color{blue!70!black}Lower peaks};

\draw[arrows=<-, thick, color=blue!20] (3,0.04)--(3.65,0.10);
\node at (3.9,0.13) {\footnotesize\color{blue!70!black}More areas in tails};

\node at (-0.5,0.4) {0.4};

\end{axis}
\end{tikzpicture}

\end{document}